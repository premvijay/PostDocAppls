\documentclass[11pt]{letter}
\usepackage[margin=1in]{geometry}

\signature{\vspace{-45pt} Premvijay Velmani,\\ Senior Research Scholar, \\ IUCAA Pune, India}
\address{Inter-University Centre for Astronomy and Astrophysics \\ Email: premv@iucaa.in; Mobile: +91-8056837468}

\begin{document}

\begin{letter}{Prof. Luisa Lucie-Smith \\ Universität Hamburg \\ Hamburg, Germany}

\opening{Dear Prof. Lucie-Smith,}

With extensive experience in analyzing cosmological simulations and designing controlled numerical experiments to model galaxy-halo connections, I am eager to apply for the postdoctoral position in your research group. Fascinated by your work using machine learning to study halo profiles and large-scale structure, I am excited about the opportunity to advance AI-driven approaches in cosmology with your group at Universität Hamburg.

Currently, at IUCAA with Prof. Aseem Paranjape, I study the interplay between galaxies and their host dark matter haloes using hydrodynamical cosmological simulations such as IllustrisTNG, EAGLE, and CAMELS. My main focus is on understanding the impacts of galactic astrophysics on dark matter halos and developing simple physical models to isolate these effects from those driven by cosmology and dark matter physics on spherically averaged density profiles. I have also performed cosmological hydrodynamical simulations and explored halos, galaxies, and the large-scale structure using tools such as GADGET, AREPO, MUSIC, ROCKSTAR, and VELOCIraptor.

Additionally, I have developed more tractable numerical experiments to study dynamical interactions between galaxies and their host dark matter haloes. While discussing one such experiment employing self-similar techniques with Prof. Susmita Adhikari, I became intrigued by the insights into halo profiles provided by your machine learning approaches.

Although I was fascinated by algorithmic approaches to learning nearly a decade ago, I initially limited my use of techniques such as deep learning neural networks to non-scientific applications. I was hesitant about their "black box" nature, fearing it might complicate building a physical picture. However, learning about interpretable techniques like glass-box models shifted my perspective, especially as I explored your work. More recently, I started collaborating on projects using machine learning for tasks like data compression, further solidifying my interest in these techniques as advanced computational statistics for cosmology. Building on this foundation, I am eager to further develop my expertise in these ML techniques and integrate with my strong expertise in simulations to understand more about galaxies and halos in the cosmos.

In direct continuation of my current research, I have well-defined plans, which aims to revolutionize the understanding and modeling of the astrophysical impacts on dark matter distribution within haloes and the large-scale structure. These long-term goals align closely with the work I hope to pursue with your group and would greatly benefit from your expertise and the computational resources available at Hamburg.

The opportunity to join your group at Hamburg Observatory is especially appealing given its vibrant research environment and emphasis on fostering interdisciplinary collaborations. I am enthusiastic about engaging with colleagues in related fields, leveraging the observatory's ties to DESY and the Quantum Universe Cluster, and contributing to your group's exploration of novel machine learning methods in cosmology.

In addition to conducting original research, I am committed to promoting diversity and inclusion in academia. As an advocate for mentoring early-career researchers, I look forward to mentoring students and fostering a collaborative, inclusive research environment. I also enjoy popularizing science by explaining elegant and intricate aspects of the Universe through accessible analogies.

Thank you for considering my application. I look forward to the opportunity to discuss our research interests and plans further.

\closing{Sincerely,}

\end{letter}

\end{document}
