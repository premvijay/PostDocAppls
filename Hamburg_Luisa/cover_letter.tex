\documentclass[11pt]{letter}
\usepackage[margin=1in]{geometry}

\signature{\vspace{-45pt} Premvijay Velmani,\\ Senior Research Scholar, \\ IUCAA Pune, India}
\address{Inter-University Centre for Astronomy and Astrophysics \\ Email: premv@iucaa.in; Mobile: +91-8056837468}

\begin{document}

\begin{letter}{Prof. Luisa Lucie-Smith \\ Universität Hamburg \\ Hamburg, Germany}

\opening{Dear Prof. Lucie-Smith,}

With extensive experience in analyzing and running cosmological simulations and designing controlled numerical experiments to model the galaxy-halo connections, I am eager to apply for this postdoctoral research position. 
Fascinated by your machine learning techniques in understanding halo profiles and large-scale structure through simulations, I am excited about the opportunity to use artificial intelligence approaches for cosmology with your new group at Hamburg.

Currently at IUCAA with Prof. Aseem Paranjape, I study the interplay between galaxies and their host dark matter haloes using hydrodynamical cosmological simulations such as IllustrisTNG, EAGLE, and CAMELS. My main focus is on understanding the impacts of galactic astrophysics on the dark matter halo and producing simple physical model that isolates these effects from other effects due to cosmology and dark matter physics on their spherically-averaged density profiles. I have also performed cosmological hydrodynamical simulation and explored halos, galaxies and the large-scale structure leveraging tools like GADGET, AREPO, MUSIC, ROCKSTAR, VELOCIraptor, etc.

Additionally, I have developed more tractable numerical experiments to study dynamical interactions between galaxies and their host dark matter haloes. While discussing one such experiment with self-similar techniques with Prof. Susmita Adhikari, I learned about some of the interesting insights about halo profiles from your work using machine learning approach. 

While I got fascinated by the algorithmic approaches to learning nearly a decade ago, I have initially limited my use of techniques such as deep learning neural networks to non-scientific applications. I was reluctant that their black box nature might make it only more difficult in building a physical picture. However, later as I learned about interpretable techniques such as glass box models I got interested in their use in cosmology, especially fascinated by your works. As I recently started working in a collaboration focused on using ML techniques such data compression, etc, I realised the enormous potential of using the ML techniques as advanced computational statistics in understanding the Universe. Building on this foundation, I am eager to further develop my expertise in these ML techniques and combine with my strong expertise in performing and analysing simulations to understand more about galaxies and haloes in the cosmos.

In direct continuation of my current research, I have well-defined long-term research plans, detailed in my research statement, which aim to make significant contributions to understanding and modeling the astrophysical impacts on dark matter distribution within haloes and the large-scale structure. This research will also greatly benefit from the computational resources available at the Hamburg and the expertise of you and your group.

The opportunity to join your group at Hamburg Observatory is especially appealing given its vibrant research environment and its emphasis on fostering interdisciplinary collaborations. I am enthusiastic about engaging with colleagues in related fields, leveraging the observatory's ties to DESY and the Quantum Universe Cluster, and contributing to your group's exploration of novel machine learning methods in cosmology.

In addition to conducting original research, I am committed to advancing the principles of diversity and inclusion in academia. As an advocate for mentoring early-career researchers, I look forward to mentoring students and fostering a collaborative and inclusive research atmosphere. I also look forward to science popularisation activaties of explaining elegant and intricate aspects of Universe through simpler analogies.

Thank you for considering my application. I look forward to the possibility of discussing my contributions to your group's cutting-edge research. 

\closing{Sincerely,}

\end{letter}

\end{document}
