Quick Questions for Dr. Lina Necib Postdoc Position
Please take some time and respond to the questions below. There is no word limit for each question, use your best judgment on how long your answers would be. There is no need to spend more than an hour answering these questions. They are meant to be quite brief.  

Please send your response as soon as you can, and if you have any questions please do not hesitate to reach out and let me know.

Prof. Lina Necib)


premv@iucaa.in Switch accounts
 
Not shared
 
Draft saved
* Indicates required question
Name 
*
Premvijay Velmani
1. List up to three skills (academic and non-academic) that make you the best match for this position?

*
i.) Strong expertise in running and analyzing cosmological simulations with and without baryonic astrophysics using specialized tools in addition to Python, C++ and shell scripting.
ii.) Experience working with observations and survey data and the use of advanced statistical techniques such as machine learning.
iii.) Ability to learn quickly anything complex and understand deeply whenever something triggers my curiosity. Fortunately, I am now passionately curious about the stellar dynamics in the Milky Way and its connection to dark matter distribution. With a general theoretical understanding of learning algorithms, I am also eager to work with specialised ML codes in this research.
2. How does this position fit into your long term goals? What skills are you hoping to gain?

*
Using simulations run with existing observational and theoretical knowledge, my main PhD research is to model and physically understand emergent phenomena. This position would further my expertise in connecting such emergent phenomena with direct observations. Also understanding the stellar dynamics and dark matter connection is aligned with my current research of understanding the interplay of baryons in galaxies with dark matter. 

Besides developing my simulation expertise further, I am hoping to gain expertise in 
dealing with large amounts of observation data, such as stellar catalogues, and extracting useful info with the help of neural networks and other advanced computational statistical techniques.
3. Tell us (briefly) about your experience working with Machine Learning, if any? You can just mention the names of algorithms you have used.

*
General background:
Almost a decade ago, I started playing with machine-learning codes like sklearn in python a hobby. The statistical theory behind such learning algorithms also provided me an intuition of our natural learning process and the generally scientific methodology of learning about the world. 

Professionally in research:
I have been passionately following the use of neural network emulators built with simulation data. Recently, I started working with data compression using techniques like Information Maximizing Neural Networks IMNN.
4. Tell us (briefly) about your experience working with cosmological simulations, if any? You can just mention the names of the codes you have used, and any simulations you ran.

*
I have performed cosmological and zoom simulations by,
-> Generating the initial conditions with Zeldovich approximation/ Lagrangian Perturnbation theory using codes namely MUSIC, MUSIC2-monofonIC and N-GenIC along with CAMB
-> Simulating with codes, GADGET, SWIFT, AREPO and some of their derivatives.

I have analysed my simulations and high-resolution simulations from large collaborations such as CAMELS, IllustrisTNG and EAGLE in following ways:
-> Finding structures with codes namely ROCKSTAR, VELOCIraptor-STF, FOF+SUBFIND to generate halo catalogues and then create merger trees.
-> Use particle data to generate density and velocity fields in both cosmological volumes and individual haloes.
-> Match the haloes between simulation with galaxies and without any baryonic astrophysics and quantify the effects of stars and gas on the dark matter distribution.
5. Tell us about a challenge you encountered in research, and how you managed it. 

*
In one of my first work, my goal was to quantify the relaxation response of dark matter halo to galactic astrophysics in simulations like IllustrisTNG. However, the common methods employed in quantifying this response seemed both noisy and failed to capture the response  information. I developed a method to quantify the response at fixed distant from the halo centres in simulations and showed that the response has a simple and more universal behaviour with this quantification than the usual methods.

Next goal was to understand the physics behind why the relaxation has a simple behaviour as a function of the halo-centric distance. Initial discussion with some experts focussed on various possibilities such as angular momentum dissipation. However, I proposed and showed with simulations that the dependence on halo-centric distance is due to the associated different timescales of response to feedback outflows from the galaxy.
6. Tell us about a challenge you encountered while mentoring, and how you managed it. 

*
Many times, while mentoring early career researchers, I feel they have a narrower view of research and, hence, they emphasize not-so-interesting results and miss out on some interesting results. In those times, I try to explain how their research is connected to a broader perspective so they can figure out what is their interesting result.
Do you have any offers with deadlines earlier than February 15th that you need us to know about?
7. Finally, what words would a good friend or colleague use to describe what it is like working with you.

*
They tell me that I make things more complicated, but I am good at understanding anything deeply and coming up with unambiguous solutions. 
I am called a man of transformation since I keep changing and evolving in various aspects.
8. Do you have any questions for us?
*
Would like to discuss more on the project.
If I could know exactly what kind of simulations we are planning to run for this project, that would help me plan my other research goals with those simulations.
