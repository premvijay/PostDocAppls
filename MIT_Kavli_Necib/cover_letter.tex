\documentclass[11pt]{letter}
\usepackage[margin=1in]{geometry}

\signature{\vspace{-45pt} Premvijay Velmani,\\ Senior Research Scholar, \\ IUCAA Pune, India}
% \address{}
\address{Inter-University Centre for Astronomy and Astrophysics \\ Email: premv@iucaa.in; Mobile: +91-8056837468}

\begin{document}

% \begin{center}
%     {\Huge \textbf{Cover Letter}} \\
%     Premvijay Velmani | Email: \texttt{premv@iucaa.in} | Mobile: +91-8056837468
% \end{center}

% \vspace{1em}

\begin{letter}{Prof. Lina Necib and members of the search committee \\ Kavli Institute for Astrophysics and Space Research \\ Massachusetts Institute of Technology, Cambridge, MA, USA}

\opening{Dear members of the search committee,}


With a strong expertise in dark matter haloes and galaxies in the cosmological simulations and further experience in working with large observational data sets and machine learning techniques, I am interested and find myself suitable for this postdoctoral associate position at the MIT Kavli Institute to work with Prof. Necib and her team. I am excited to start immediately with the primary research project of understanding dark matter and galactic dynamics by infering GAIA and LSST observations through simulations; and I also have other research plans and ideas that will be interesting to explore with Prof. Necib and her group.

In my current research at the Inter-University Centre for Astronomy and Astrophysics, my main focus is on understanding and modelling the effect of astrophysical processes on the dark matter within haloes. This primarily involves analysing state-of-the-art cosmological simulations with galaxies such as IllustrisTNG, EAGLE and CAMELS. While exploring this, I have also developed more tractable numerical experiments to study the synamical interation between the galaxies and their host dark matter haloes. 

As part of my research, I have performed cosmological simulations producing galaxies, dark matter haloes and other large scale cosmological quantities using codes such as GADGET, MUSIC, ROCKSTAR, VELOCIraptor, etc. Additionally, I have worked on a mini project with Prof. Hector Marin on inferring cosmological parameters from eBOSS and mock DESI datasets. I am currently part of a data science collaboration focused on employing machine learning techniques for cosmological data compression and inference. With these experiences, I am sure I can immediately start working on the primary project with the Prof. Necib and her group.

In direct continuation of my current research, I aim to make significant research contributions in understanding and modelling the astrophysical impacts on the dark matter haloes with well-defined research plans as detailed in my research statement. This research will not only greatly benefit from the resources available at the Kavli Institute but also from the expertise of the Prof.Necib's team.

Thank you for considering my application. I look forward to the opportunity to contribute to the group's efforts and would be delighted to discuss my expertise further.

\closing{Sincerely,}

\end{letter}
\end{document}
