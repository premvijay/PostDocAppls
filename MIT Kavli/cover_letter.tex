\documentclass[11pt]{letter}
\usepackage[margin=1in]{geometry}
\usepackage{hyperref}

\signature{\vspace{-45pt} Premvijay Velmani,\\ Senior Research Scholar, \\ IUCAA Pune, India}
\address{Inter-University Centre for Astronomy and Astrophysics \\ Email: premv@iucaa.in; Mobile: +91-8056837468}

\begin{document}

\begin{letter}{Search Committee \\ MIT Kavli Institute for Astrophysics and Space Research \\ Massachusetts Institute of Technology \\ Cambridge, MA, USA}

\opening{Dear Members of the Search Committee,}

I am writing to express my interest in the MIT Kavli Postdoctoral Fellowship in Astrophysics. My research bridges galactic astrophysics with cosmology and dark matter physics, utilizing full hydrodynamic cosmological simulations such as IllustrisTNG, EAGLE, and CAMELS to study the impact of galactic astrophysical processes on dark matter haloes. In parallel, I develop controlled numerical experiments to investigate galaxy formation and its interplay with dark matter haloes in simplified environments. These complementary approaches have equipped me with the expertise and independence to design and lead innovative research projects. \textbf{I have a clearly-formulated research plan to construct a physical description of the response of dark matter haloes to galactic astrophysical processes, aiming to significantly enhance our ability to infer cosmology and dark matter physics from observations.}

Currently, I am a Senior Research Scholar at IUCAA, where I have submitted my PhD thesis under the supervision of Prof. Aseem Paranjape. My thesis focuses on the astrophysical effects of galaxy formation on dark matter haloes, emphasizing changes in radial density profiles, which are crucial to understanding observables such as rotation curves. While my primary focus has been analyzing large-scale cosmological simulations, I also develop semi-numerical experiments to study specific aspects of galaxy-halo interactions in a controlled setting. My technical expertise spans running cosmological hydrodynamic simulations using tools such as GADGET, AREPO, and MUSIC, as well as analyzing halos, galaxies, and the large-scale structure with ROCKSTAR and VELOCIraptor. I am also engaged in collaborative efforts utilizing advanced statistical and machine learning techniques to analyze the wealth of data produced by cosmological surveys. I have did cosmological inferences from surveyss like eBOSS and mock DESI data.

Looking forward, my research will continue to probe galaxy-halo interactions across cosmic time, using a framework I developed to quantify the time-correlated effects of astrophysical feedback, particularly AGN-driven outflows, on the evolution of dark matter haloes. This research has direct implications for cosmological inference from large surveys, particularly through weak lensing and clustering analyses. Furthermore, I am eager to expand my work into areas such as cross-survey analyses and machine-learning approaches to bridge simulations and observations. MKI's unique combination of theoretical, observational, and computational expertise provides an ideal environment for advancing this research.

The opportunity to collaborate within MKI’s rich intellectual environment and to engage with its cutting-edge observational facilities and computational resources excites me greatly. I am particularly inspired by MKI’s institutional involvement in upcoming surveys and missions like CHIME and the Vera Rubin Observatory, where my expertise in dark matter physics and galaxy formation modeling can contribute meaningfully. The emphasis on fostering inclusion and belonging at MKI resonates strongly with my commitment to creating diverse and supportive academic environments through mentoring and outreach.

Thank you for considering my application. I have included my CV, research statement, and publication list. I look forward to discussing my potential contributions to the MIT Kavli Institute and its vibrant astrophysics community.

\closing{Sincerely,}

\end{letter}

\end{document}
