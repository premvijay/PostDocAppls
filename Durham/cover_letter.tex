\documentclass[10pt]{letter}
\usepackage[margin=1.25in]{geometry}
\usepackage{hyperref}

\address{Premvijay Velmani, Senior Research Scholar, \\Inter-University Centre for Astronomy and Astrophysics \\ Email: premv@iucaa.in; Mobile: +91-8056837468}
\date{December 16, 2024}

\begin{document}

\begin{letter}{Search Committee \\ Institute for Computational Cosmology \\ Durham University, UK}

\opening{Dear Members of the Search Committee,}
    
I am writing to express my interest in the advertised postdoctoral positions at the Institute for Computational Cosmology (ICC), Durham University. My research bridges galactic astrophysics with cosmology and dark matter physics, utilising full hydrodynamic cosmological simulations such as IllustrisTNG, EAGLE, and CAMELS to study the impact of galactic astrophysical processes on dark matter haloes. In parallel, I develop controlled numerical experiments to investigate galaxy formation and its interplay with dark matter haloes in simplified environments. I have a clear research plan to construct a physical description of the response of dark matter haloes to galactic astrophysical processes, aiming to significantly enhance our ability to infer cosmology and dark matter physics from observations.
    
I see a strong synergy between my research interests and ongoing work at ICC. Dr. Sownak Bose’s research, especially those with Dr. Sorini on the galaxy-halo connection in simulations, is directly connected to my work on halo relaxation responses and baryonic feedback effects. Also, Dr. Fragkoudi’s focus on galaxy dynamics and their interplay with the cosmological context resonates with my efforts to model the impact of astrophysical feedback on halo shapes and radial profiles in isolated and cosmological settings. Additionally, my research interests closely align with those of several other groups at ICC, including Prof. Adrian Jenkins, Prof. Carlton Baugh, Prof. Shaun Cole, and Prof. Carlos Frenk.
    
During my PhD at IUCAA, I gained expertise in running and analysing cosmological hydrodynamical simulations with codes like GADGET, AREPO, ROCKSTAR, VELOCIraptor, MUSIC, etc. Through state-of-the-art simulations like IllustrisTNG and CAMELS, I have identified some key features in the dynamical response of dark matter distribution in haloes to galactic processes, such as feedback. Besides such full simulations, I develop more tractable numerical experiments of the interplay of galaxy-halo evolution. I have also worked with observation data, such as inferring cosmology from eBOSS and mock DESI data. I am currently in a collaboration focused on applying advanced statistical and machine learning techniques. My technical skills in running and analysing simulations, combined with an interest in improving simulation codes with modern computational methods, will allow me to contribute to ICC’s world-class research environment effectively.
    
Durham’s leadership in cosmological research and its extensive computational resources, including the “Cosmology Machine,” make it an ideal environment to pursue my research goals. I am particularly enthusiastic about collaborating on projects that leverage ICC’s strength in simulations to advance our understanding of galaxy-halo connections and the nonlinear evolution of cosmic structures.
    
Thank you for considering my application. I have included my CV, bibliography, and research statement, and I would be delighted to discuss how my research aligns with ICC’s scientific vision.

\end{letter}

\end{document}
