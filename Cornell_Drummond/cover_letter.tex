\documentclass[11pt]{letter}
\usepackage[left=1in,top=.5in,bottom=0.5in,right=1.2in]{geometry}
\geometry{paper=letterpaper}

\signature{\vspace{-45pt} Premvijay Velmani,\\ Senior Research Scholar, \\ IUCAA Pune, India}
\address{Inter-University Centre for Astronomy and Astrophysics \\ Email: premv@iucaa.in; Mobile: +91-8056837468}

\begin{document}

\begin{letter}{Prof. Drummond Fielding and members of the search committee \\ Cornell Center for Astrophysics and Planetary Science \\ Cornell University, Ithaca, NY, USA}

\opening{Dear members of the search committee,}

With extensive experience in analyzing and running cosmological simulations and designing controlled numerical experiments, I am eager to apply for this postdoctoral research position in theoretical astrophysics at the Cornell Center for Astrophysics and Planetary Science. I am particularly excited by the opportunity to work with Prof. Drummond Fielding's group in understanding the rich and elegant astrophysical processes such as galactic winds, turbulence, and their effects on the CGM. I also have specific research plans closely connected to this area of study, leveraging the resources available at Cornell and expertise of the group.

My current research at the Inter-University Centre for Astronomy and Astrophysics focuses on understanding the interplay between astrophysical processes and dark matter haloes. I have extensively analyzed hydrodynamical cosmological simulations such as IllustrisTNG, EAGLE, and CAMELS, investigating the role of various astrophysical processes such as feedback mechanisms in shaping dark matter distributions within haloes. Additionally, I have developed and performed controlled numerical experiments to study dynamical interactions between galaxies and their host dark matter haloes, leveraging tools like GADGET, MUSIC, and VELOCIraptor.  

In the course of my PhD research, I have gained expertise in running and analyzing hydrodynamical simulations primarily in cosmological volumes, leveraging tools like GADGET, AREPO, SWIFT, etc. While I have some experience with Athena++, I am eager to further develop my expertise with it and deploy it  alongside codes like AthenaK and GIZMO to contribute to the group's numerical simulation efforts. I am also keen to explore new physical processes such as cosmic ray transport and their implications for galaxy formation and feedback mechanisms, which will be directly useful for my proposed research goals while aligning with the group's research themes.

In continuation of my current work, I aim to design more tractable numerical experiments that probe the coupling between different feedback channels (e.g., AGN and stellar feedback) and their effects on galactic winds and their host dark matter haloes.  

I am confident that my expertise and research plans will make a meaningful contribution to the group's efforts. I am excited about the collaboration opportunities offered by this position, both within Cornell and with international teams to explore my long-term research goals.

Thank you for considering my application. I look forward to the opportunity to discuss my expertise and research ideas further.  

\closing{Sincerely,}

\end{letter}

\end{document}
