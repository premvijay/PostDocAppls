\documentclass[11pt]{letter}
\usepackage[margin=1in]{geometry}
\usepackage{hyperref}

\signature{\vspace{-45pt} Premvijay Velmani,\\ Senior Research Scholar, \\ IUCAA Pune, India}
\address{Inter-University Centre for Astronomy and Astrophysics \\ Email: premv@iucaa.in; Mobile: +91-8056837468}

\begin{document}

\begin{letter}{Search Committee \\ Beus Prize Fellowship \\ School of Earth and Space Exploration \\ Arizona State University \\ Tempe, AZ, USA}

\opening{Dear Members of the Search Committee,}

I am writing to express my interest in the Beus Prize Fellowship at Arizona State University. My research bridges galactic astrophysics and cosmology, leveraging full hydrodynamic cosmological simulations such as IllustrisTNG, EAGLE, and CAMELS to investigate how astrophysical processes in galaxies impact the nature and evolution of dark matter haloes. In addition toperforming and analyzing such cosmological simulations, I develop controlled numerical experiments to study the formation of galaxies and their interaction with their host dark matter haloes. These efforts have equipped me with the expertise and independence necessary to design and lead innovative research projects that align with the Beus Center's vibrant academic environment. I have also formulated a well-defined research plan to construct a physical description of the response of dark matter haloes to galactic astrophysics, aiming to significantly enhance our inferences about cosmology and dark matter physics from observations.

Currently, I am a Senior Research Scholar at IUCAA, where I have submitted my PhD thesis under the supervision of Prof. Aseem Paranjape. My thesis focuses on the astrophysical effects of galaxy formation on dark matter haloes, emphasizing changes in radial density profiles, relevant to observations such as rotation curves. While this primarily involves analyzing state-of-the-art cosmological simulations that produce realistic galaxies, I also develop tractable semi-numerical experiments to study galaxy-halo interactions in a more controlled manner. Additionally, I am engaged in a data science collaboration utilizing advanced statistical techniques, such as machine learning, to extract insights from the wealth of data produced by cosmological surveys.

Building on this foundation, my proposed research seeks to establish a cohesive model for galaxy-halo interactions over cosmic time. A key goal is to quantitatively connect astrophysical feedback, particularly AGN-driven outflows, to the structural evolution of dark matter haloes through my time-correlated framework. Leveraging the Arizona State's computational resources and collaborations, I plan to advance this effort significantly through both cosmological simulations and controlled numerical experiments. My prior experience with cosmological inferences using galaxy surveys, such as eBOSS and mock DESI data in collaboration with Prof. Hector Marin, further motivates me to contribute to the Centre's observational efforts.

Arizona State University's focus on interdisciplinary approaches and its collaborative environment make it an ideal setting to expand this work. I am particularly excited about the opportunity to collaborate with SESE researchers and affiliates of the Beus Center for Cosmic Foundations to address fundamental questions in galaxy formation and evolution.

The Beus Center's commitment to fostering impactful research and societal engagement resonates with my interests in mentoring students and organizing public lectures on astrophysics. I am deeply committed to fostering an inclusive academic environment and am eager to contribute to the Center's vision of making intricate elegance of the Universe accessible to diverse audiences.

Thank you for considering my application. I have included my CV, publication list, and a detailed description of my research. I look forward to the opportunity to contribute to the Center's research endeavors and would welcome the chance to discuss my plans in more detail.

\closing{Sincerely,}

\end{letter}

\end{document}
