\documentclass[11pt]{letter}
\usepackage[margin=1in]{geometry}

\signature{\vspace{-45pt} Premvijay Velmani,\\ Senior Research Scholar, \\ IUCAA Pune, India}
\address{Inter-University Centre for Astronomy and Astrophysics \\ Email: premv@iucaa.in; Mobile: +91-8056837468}

\begin{document}

\begin{letter}{Prof. Enrique Lopez-Rodriguez and members of the search committee \\ Department of Physics and Astronomy \\ University of South Carolina, Columbia, SC, USA}

\opening{Dear members of the search committee,}

I am writing to express my interest in the postdoctoral research position in the Extragalactic Magnetism Group at the University of South Carolina. My expertise in performing and analyzing hydrodynamical cosmological simulations, combined with my interest in understanding the evolution of magnetic fields and their interplay with baryonic processes, aligns closely with the goals of this position. I am particularly excited by the opportunity to contribute to the group's efforts in generating mock observations through cosmological MHD simulations. Furthermore, studying the properties of dust and cosmic rays closely complements my ongoing research into the astrophysical impacts on dark matter haloes.

During my PhD at IUCAA, I extensively analyzed cosmological hydrodynamical simulations (such as IllustrisTNG, EAGLE, and CAMELS) to investigate the effects of galactic astrophysical processes on the evolution of dark matter haloes. Additionally, I designed and performed controlled numerical experiments to model the dynamical interactions between galaxies and their host dark matter haloes.




As part of my research, I have conducted cosmological simulations that produce galaxies, dark matter haloes, and other large-scale cosmological quantities using codes such as GADGET, MUSIC, ROCKSTAR, and VELOCIraptor. I also collaborated with Prof. Hector Marin on a mini-project inferring cosmological parameters from eBOSS and generating mock DESI datasets. Building on this foundation, I am eager to further develop my expertise in performing high-resolution cosmological simulations, analyzing the properties of dust and cosmic rays in galaxies and the CGM, and investigating their interaction with magnetic fields. I am particularly excited about generating high-quality mock observations for radio (e.g., SKA) and infrared (e.g., JWST) facilities.

In direct continuation of my current research, I have well-defined long-term research plans, detailed in my research statement, which aim to make significant contributions to understanding and modeling the astrophysical impacts on dark matter haloes. The work I propose to undertake with Prof. Rodriguez, exploring the properties of dust and magnetic fields, will have direct applications to my long-term research goals in understanding their impacts on dark matter haloes.

In addition to contributing to the group's research efforts, I look forward to mentoring students by providing them with engaging and focused research projects. I am also enthusiastic about collaborating with other members of the Extragalactic Magnetism Group and contributing to the development of grant proposals and observing proposals targeting key research questions.

Thank you for considering my application. I am excited about the opportunity to bring my expertise in cosmological simulations to your group and to contribute meaningfully to its research endeavors. I look forward to discussing my expertise and research plans further.

\closing{Sincerely,}

\end{letter}

\end{document}