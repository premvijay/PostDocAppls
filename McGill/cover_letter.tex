\documentclass[11pt]{letter}
\usepackage[margin=1in]{geometry}
\usepackage{hyperref}

\signature{\vspace{-45pt} Premvijay Velmani,\\ Senior Research Scholar, \\ IUCAA Pune, India}
\address{Inter-University Centre for Astronomy and Astrophysics \\ Email: premv@iucaa.in; Mobile: +91-8056837468}

\begin{document}

\begin{letter}{Search Committee \\ Trottier Space Institute \\ McGill University \\ Montreal, QC, Canada}

\opening{Dear Members of the Search Committee,}

I am writing to express my interest in the Trottier Postdoctoral Fellowship at the Trottier Space Institute (TSI), McGill University. My research bridges galactic astrophysics with cosmology and dark matter physics, utilizing full hydrodynamic cosmological simulations such as IllustrisTNG, EAGLE, and CAMELS to study the impact of galactic astrophysical processes on dark matter haloes. In parallel, I develop controlled numerical experiments to investigate galaxy formation and its interplay with dark matter haloes in simplified environments. These complementary approaches have equipped me with the expertise and independence to design and lead innovative research projects. \textbf{I have a clearly-formulated research plan to construct a physical description of the response of dark matter haloes to galactic astrophysical processes, aiming to significantly enhance our ability to infer cosmology and dark matter physics from observations.}

Currently, I am a Senior Research Scholar at IUCAA, where I have submitted my PhD thesis under the supervision of Prof. Aseem Paranjape. My thesis focuses on the astrophysical effects of galaxy formation on dark matter haloes, emphasizing changes in radial density profiles, which are crucial to understanding observables such as rotation curves. While my primary focus has been analyzing large-scale cosmological simulations, I also develop semi-numerical experiments to study specific aspects of galaxy-halo interactions in a controlled setting. My technical expertise spans running cosmological hydrodynamic simulations using tools such as GADGET, AREPO, and MUSIC, as well as analyzing halos, galaxies, and the large-scale structure with ROCKSTAR and VELOCIraptor. I am also engaged in collaborative efforts utilizing advanced statistical and machine learning techniques to analyze the wealth of data produced by cosmological surveys. I have did cosmological inferences from surveyss like eBOSS and mock DESI data.

Looking forward, my research will continue to probe galaxy-halo interactions across cosmic time, using a framework I developed to quantify the time-correlated effects of astrophysical feedback, particularly AGN-driven outflows, on the evolution of dark matter haloes. McGill's strong computational and observational astrophysics programs provide an ideal platform to integrate cosmological simulations with controlled numerical experiments to achieve this goal. Additionally, TSI's collaborative environment aligns with my aim to expand this research and apply it to interpret observational data from large-scale surveys.

The interdisciplinary and inclusive community at TSI, coupled with its emphasis on addressing fundamental questions in astrophysics and cosmology, makes it an ideal environment for advancing my research. I am especially excited about collaborating with dark matter physics and cosmology research groups at TSI. By leveraging the physical model of dark matter halo response to galaxies, I aim to significantly improve our inference about the dark matter physics and the cosmos in these collaborations.

Beyond research, I am committed to promoting diversity and equity in academia. By mentoring early-career researchers and engaging in outreach activities, I aim to contribute to TSI's mission of fostering an inclusive scientific community.

Thank you for considering my application. I have included my CV, research statement, and publication list. I look forward to the opportunity to contribute to TSI's vibrant academic environment and ambitious research initiatives.

\closing{Sincerely,}

\end{letter}

\end{document}
