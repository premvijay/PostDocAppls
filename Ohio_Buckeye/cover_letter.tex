\documentclass[11pt]{letter}
\usepackage[margin=1in]{geometry}

\signature{\vspace{-45pt} Premvijay Velmani,\\ Senior Research Scholar, \\ IUCAA Pune, India}
\address{Inter-University Centre for Astronomy and Astrophysics \\ Email: premv@iucaa.in; Mobile: +91-8056837468}

\begin{document}

\begin{letter}{Search Committee \\ Department of Astronomy \\ The Ohio State University \\ Columbus, OH, USA}

\opening{Dear Members of the Search Committee,}

I am writing to express my interest in the Postdoctoral Scholar position at The Ohio State University. My research bridges galaxy formation and cosmology, leveraging full hydrodynamic cosmological simulations such as IllustrisTNG, EAGLE, and CAMELS to investigate how astrophysical processes impact dark matter halo properties. This work, which synthesizes data-driven and theoretical approaches, has equipped me with the expertise and independence necessary to design and lead innovative research projects that align with Ohio State’s vibrant academic environment.

Currently, I am a Senior Research Scholar at IUCAA, where I am completing my PhD under the supervision of Prof. Aseem Paranjape. My thesis explores the astrophysical effects of galaxy formation on dark matter haloes, analyzing their density profiles, power spectrum, and relaxation mechanisms in simulations. In addition, I have developed idealized numerical experiments to study galaxy-halo interactions, employing techniques such as spherical shells to isolate dynamical responses to galaxy evolution processes. These controlled experiments complement my large-scale simulations and provide a foundation for my future work.

Building on this foundation, my proposed research aims to provide a cohesive model for understanding galaxy-halo interactions across time. A primary goal is to quantitatively connect astrophysical feedback, particularly AGN-driven outflows, to the relaxation and structural evolution of dark matter haloes. I also plan to develop a time-correlated framework for modeling these interactions, bridging halo-centric models with galaxy-centric observational proxies. Leveraging Ohio State’s computational resources and collaborations, I hope to expand these efforts into models that inform both theoretical studies and large-scale surveys such as DESI. This plan aligns with your department’s strengths in cosmology, large-scale structure, and galaxy formation.

Ohio State’s extensive observational and computational resources, including ties to DESI and the Ohio Supercomputing Center, are uniquely suited to support my proposed work. The department’s collaborative research environment, particularly the interdisciplinary opportunities provided by CCAPP, would also enrich my development as an independent researcher. I am particularly excited about contributing to the department's legacy of leadership in both theoretical and observational cosmology.

Beyond research, I am passionate about fostering a diverse and inclusive academic environment. I have mentored undergraduate and master’s students in astrophysics and actively participated in outreach activities aimed at making science accessible to broader audiences. As part of the Ohio State community, I look forward to engaging in similar efforts and mentoring early-career researchers.

Thank you for considering my application. I have included my CV, publication list, and a detailed description of my past and proposed research. I look forward to the possibility of contributing to the department’s research endeavors and would welcome the opportunity to discuss my plans in more detail.

\closing{Sincerely,}

\end{letter}

\end{document}
