\documentclass[11pt]{letter}
\usepackage[margin=1in]{geometry}

\signature{\vspace{-45pt} Premvijay Velmani,\\ Senior Research Scholar, \\ IUCAA Pune, India}
\address{Inter-University Centre for Astronomy and Astrophysics \\ Email: premv@iucaa.in; Mobile: +91-8056837468}

\begin{document}

\begin{letter}{Search Committee \\ Department of Astronomy \\ The Ohio State University \\ Columbus, OH, USA}

\opening{Dear Members of the Search Committee,}

I am writing to express my interest in the Buckeye Postdoctoral position at The Ohio State University. My research bridges galactic astrophysics and cosmology, leveraging full hydrodynamic cosmological simulations such as IllustrisTNG, EAGLE, and CAMELS to investigate how astrophysical processes in galaxies impact the nature of evolution of dark matter haloes. In addition to performing and analyzing such cosmological simulations, I also develop more controlled numerical experiments of the formation of galaxies and their interaction with there host dark matter haloes. This has provided me with the expertise and independence necessary to design and lead innovative research projects that align with Ohio State's vibrant academic environment. Now I have a well-defined research plan that aims to build an entirely physical description of the response of dark matter haloes to galactic astrophysics and significantly improve our inferences about cosmology and dark matter physics from their observations.

Currently, I am a Senior Research Scholar at IUCAA, where I have submitted my PhD thesis under the supervision of Prof. Aseem Paranjape. My thesis explores the astrophysical effects of galaxy formation and evolution on dark matter haloes, focusing on their radial density profiles which is relvant for observations such as rotation curves, etc. While this primary involves analyzing state-of-the-art cosmological simulations producing realistic galaxies, I also develop more tractable semi-numerical experiments to study galaxy-halo interactions. Currently, I am also working in a data science collaboration focused on using advance statical techniques such as machine leaning in understanding the Universe from abundant data produced by observations. 

Building on this foundation, my proposed research aims to provide a cohesive model for understanding galaxy-halo interactions across time. A primary goal is to quantitatively connect astrophysical feedback, particularly AGN-driven outflows, to the relaxation and structural evolution of dark matter haloes using my time-correlated framework leveraging Ohio State's computational resources and collaborations. This plan aligns with your department's strengths in cosmology, large-scale structure, and galaxy formation. Having done cosmological inferences from large galaxy surveys such as the eBOSS and with mock DESI data with Prof. Hector Marin, I am also excited about the Ohio State's ties with DESI.

Ohio State's extensive observational and computational resources, including ties to DESI and the Ohio Supercomputing Center, are uniquely suited to support my proposed work. The department's collaborative research environment, particularly the interdisciplinary opportunities provided by CCAPP, would also enrich my development as an independent researcher. I am particularly excited about contributing to the department's legacy of leadership in both theoretical and observational cosmology.

Beyond research, I am passionate about fostering a diverse and inclusive academic environment, mentoring students in astrophysics and actively participate in outreach activities aimed at making science accessible to broader audiences. As part of the Ohio State community, I look forward to engaging in similar efforts and mentoring early-career researchers.

Thank you for considering my application. I have included my CV, publication list, and a detailed description of my past and proposed research. I look forward to the possibility of contributing to the department's research endeavors and would welcome the opportunity to discuss my plans in more detail.

\closing{Sincerely,}

\end{letter}

\end{document}
