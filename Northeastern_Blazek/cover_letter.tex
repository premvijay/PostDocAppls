\documentclass[11pt]{letter}
\usepackage[margin=1in]{geometry}
\usepackage{hyperref}

\signature{\vspace{-45pt} Premvijay Velmani,\\ Senior Research Scholar, \\ IUCAA Pune, India}
\address{Inter-University Centre for Astronomy and Astrophysics \\ Email: premv@iucaa.in; Mobile: +91-8056837468}

\begin{document}

\begin{letter}{Search Committee \\ Department of Physics \\ Northeastern University \\ Boston, MA, USA}

\opening{Dear Members of the Search Committee,}

I am writing to express my interest in the postdoctoral researcher position in cosmology at Northeastern University. My research bridges cosmology and galactic astrophysics, with a focus on utilizing cosmological simulations to study the interplay between dark matter haloes and astrophysical processes. I am particularly excited about the opportunity to work with Prof. Jonathan Blazek and the cosmology group at Northeastern, where my expertise in simulations and cosmological inference can complement the observational and numerical work conducted within the group.

Currently, I am a Senior Research Scholar at IUCAA, where I have submitted my PhD thesis under the supervision of Prof. Aseem Paranjape. My thesis focuses on the dynamical response of dark matter haloes to galaxy formation, with implications for halo shapes and structure, which are key observables in weak gravitational lensing studies. I have developed a physically motivated framework to model these effects, aiming to bridge theoretical modeling with observational cosmology. My research builds on state-of-the-art hydrodynamic cosmological simulations, such as IllustrisTNG, EAGLE, and CAMELS, and involves developing controlled numerical experiments to probe specific aspects of galaxy-halo interactions.

In addition to my simulation work, I have collaborated with Prof. Hector Marin on cosmological inference using eBOSS and mock DESI data, gaining valuable experience in observational cosmology and statistical analysis. These projects have honed my ability to connect theoretical models to data, a skill directly relevant to weak lensing, galaxy clustering, and combined-probe analyses. I am eager to apply this experience to the research efforts at Northeastern, where I see strong synergy between my work on galaxy-halo connections and the analytic and numerical modeling approaches employed by Prof. Blazek.

Looking forward, my research proposal aligns well with the goals of Northeastern’s cosmology group. I have a clearly-formulated research plan to construct a physical description of the response of dark matter haloes to galactic astrophysical processes, aiming to significantly enhance our ability to infer cosmology and dark matter physics from observations. I aim to extend my work on dark matter halo response by integrating weak lensing data to constrain halo shapes and structure. This effort will benefit from the combined-probe approaches pursued by Prof. Blazek, leveraging insights from lensing and clustering to inform the physical modeling of galaxy-halo interactions. I am particularly excited about the opportunity to contribute to the Vera Rubin Observatory’s Dark Energy Science Collaboration and to explore cross-survey analyses involving lensing and other probes of large-scale structure.

Northeastern’s vibrant research environment and interdisciplinary collaborations, particularly through the NSF AI Institute for Artificial Intelligence and Fundamental Interactions, make it an ideal setting to advance my research. I look forward to contributing to this dynamic community, bringing my experience in simulations, numerical modeling, and cosmological inference to tackle fundamental questions in cosmology.

Thank you for considering my application. I have included my CV, research statement, and publication list, and I have arranged for three reference letters to be submitted through Academic Jobs Online. I look forward to the opportunity to contribute to Northeastern’s cosmology program and to discuss my potential role in its research initiatives.

\closing{Sincerely,}

\end{letter}

\end{document}
