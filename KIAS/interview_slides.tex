\documentclass{beamer}
\usepackage[utf8]{inputenc}
\usepackage{graphicx}
\usepackage{hyperref}
\usepackage{amsmath}
\usepackage{tikz}
\usepackage{multicol}
\usetheme{Boadilla}

\title[Research Overview]{Research Overview and Future Directions}
\author[PremVijay Velmani]{Premvijay Velmani \\\texttt{premv@iucaa.in}}
\institute[IUCAA]{Inter-University Centre for Astronomy and Astrophysics (IUCAA)}
\date[KIAS Interview]{January 17, 2025}

\begin{document}

% Title Slide
\begin{frame}
    \titlepage
\end{frame}

% Slide 1: Introduction
\begin{frame}{Introduction}
    \begin{itemize}
        \item Senior Research Scholar at IUCAA, Pune.
        \item Expertise in cosmological simulations and theoretical modeling.
        \item PhD Thesis submitted: "Interplay " Thesis approved by examiners and awaiting defence.
        \item Current position ends in July 2025.
    \end{itemize}
\end{frame}

% Slide 2: Past Research - Overview
\begin{frame}{Past Research: Overview}
    \begin{itemize}
        \item Focused on dark matter halo dynamics in cosmological simulations.
        \item Studied halo relaxation responses using \textbf{IllustrisTNG, EAGLE, and CAMELS}.
        \item Developed self-similar models for galaxy formation and tested against simulations.
        \item Analyzed astrophysical feedback processes, especially \textbf{AGN feedback impacts} on halo structure.
    \end{itemize}
\end{frame}

% Slide 3: Key Contributions
\begin{frame}{Key Contributions}
    \begin{itemize}
        \item Established connections between galaxy formation processes and halo relaxation timescales.
        \item Quantified the impact of baryonic physics on dark matter distribution.
        \item Improved time-correlation analysis techniques for astrophysical feedback processes.
        \item Collaborated on large-scale survey mock simulations for EUCLID and DESI.
    \end{itemize}
\end{frame}

% Slide 4: Proposed Research Directions
\begin{frame}{Proposed Research Directions}
    \begin{itemize}
        \item Enhance time-correlation methods to study baryonic feedback in dark matter haloes.
        \item Develop and implement \textbf{cosmological MHD simulations} for better theoretical modeling.
        \item Investigate galaxy-halo connections with a focus on AGN-driven relaxation processes.
        \item Run mock simulations to improve interpretations of observations from large-scale surveys.
    \end{itemize}
\end{frame}

% Slide 5: Why KIAS?
\begin{frame}{Why KIAS?}
    \begin{itemize}
        \item Aligns with my research focus on dark matter haloes and galaxy formation.
        \item Opportunity to collaborate with experts in theoretical and computational astrophysics.
        \item Access to resources and a vibrant academic environment.
        \item Chance to contribute to cutting-edge cosmological research.
    \end{itemize}
\end{frame}

% Slide 6: Summary
\begin{frame}{Summary}
    \begin{itemize}
        \item Extensive experience in cosmological simulations and astrophysical modeling.
        \item Strong focus on understanding baryonic impacts on dark matter dynamics.
        \item Excited to contribute to KIAS's research goals while pursuing innovative directions.
        \item Looking forward to discussing my research and ideas further.
    \end{itemize}
\end{frame}

% Slide 7: Thank You
\begin{frame}{Thank You!}
    \begin{center}
        \textbf{Questions?}\\[1em]
        \includegraphics[width=0.2\textwidth]{example-image-a} \\
        \texttt{premv@iucaa.in}
    \end{center}
\end{frame}

\end{document}
