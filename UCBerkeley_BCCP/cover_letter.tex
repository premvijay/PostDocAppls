\documentclass[11pt]{letter}
\usepackage[margin=1in]{geometry}
\usepackage{hyperref}

\signature{\vspace{-45pt} Premvijay Velmani,\\ Senior Research Scholar, \\ IUCAA Pune, India}
\address{Inter-University Centre for Astronomy and Astrophysics \\ Email: premv@iucaa.in; Mobile: +91-8056837468}

\begin{document}

\begin{letter}{Search Committee \\ Berkeley Center for Cosmological Physics \\ University of California, Berkeley \\ Berkeley, CA, USA}

\opening{Dear Members of the Search Committee,}

I am writing to express my interest in the BCCP Postdoctoral Fellowship at the University of California, Berkeley. My research bridges galactic astrophysics and cosmology, utilizing full hydrodynamic cosmological simulations such as IllustrisTNG, EAGLE, and CAMELS to explore the impact of galactic astrophysical processes on dark matter haloes. In addition to performing and analyzing such cosmological simulations, I develop controlled numerical experiments to investigate the formation and evolution of galaxies and their interplay with host dark matter haloes. These efforts reflect my commitment to advancing our understanding of cosmological structures, aligning closely with the Berkeley Center for Cosmological Physics' mission to address fundamental questions in cosmology. These efforts have equipped me with the expertise and independence necessary to design and lead innovative research projects that align with the Berkeley Center for Cosmological Physics' mission to address fundamental questions in cosmology. \textbf{I have a clearly-formulated research plan to construct a physical description of the response of dark matter haloes to galactic astrophysics, aiming to significantly enhance our inferences about cosmology and dark matter physics from observations.}

Currently, I am a Senior Research Scholar at IUCAA, where I have submitted my PhD thesis under the supervision of Prof. Aseem Paranjape. My thesis focuses on the astrophysical effects of galaxy formation on dark matter haloes, emphasizing changes in radial density profiles, relevant to observations such as rotation curves. While this primarily involves analyzing state-of-the-art cosmological simulations that produce realistic galaxies, I also develop tractable semi-numerical experiments to study galaxy-halo interactions in a more controlled manner. I have also performed cosmological hydrodynamical simulations and explored halos, galaxies, and the large-scale structure using tools such as GADGET, AREPO, MUSIC, ROCKSTAR, and VELOCIraptor. Additionally, I am engaged in a data science collaboration utilizing advanced statistical techniques, such as machine learning, to extract insights from the wealth of data produced by cosmological surveys.

Looking forward, my proposed research seeks to establish a cohesive model for galaxy-halo interactions over cosmic time. A key goal is to quantitatively connect astrophysical feedback, particularly AGN-driven outflows, to the evolution of dark matter haloes through my time-correlated framework. Leveraging the Berkeley Center's computational resources and collaborations, I plan to advance this effort significantly through both cosmological simulations and controlled numerical experiments. My prior experience with cosmological inferences using galaxy surveys, such as eBOSS and mock DESI data in collaboration with Prof. Hector Marin, further motivates me to contribute to the Centre's observational efforts.

The Berkeley Center’s collaborative and interdisciplinary environment offers an unparalleled opportunity to develop this work further, particularly through interactions with researchers in the Astronomy and Physics Departments and at LBNL.

UC Berkeley's unique combination of observational programs, data science expertise, and computational resources provides an ideal platform to expand my research. I am particularly drawn to the Center's emphasis on fostering independent research while promoting collaboration across experimental, observational, and theoretical cosmology. I look forward to contributing to this vibrant academic community, both through research and mentoring.

In addition to my scientific pursuits, I am committed to promoting diversity and inclusivity in academia. Through mentoring early-career researchers and engaging in outreach activities, I am eager to contribute to the Center's vision of making intricate elegance of the Universe accessible to diverse audiences.

Thank you for considering my application. I have included my CV, research statement, and publication list. I look forward to discussing my potential contributions to the BCCP and its research initiatives.

\closing{Sincerely,}

\end{letter}

\end{document}
