\documentclass[11pt]{letter}
\usepackage[margin=1in]{geometry}
\usepackage{hyperref}

\signature{\vspace{-45pt} Premvijay Velmani,\\ Senior Research Scholar, \\ IUCAA Pune, India}
\address{Inter-University Centre for Astronomy and Astrophysics \\ Email: premv@iucaa.in; Mobile: +91-8056837468}

\begin{document}

\begin{letter}{Search Committee \\ Neil Gehrels Prize Postdoctoral Fellowship \\ Joint Space-Science Institute \\ University of Maryland, College Park, MD, USA}

\opening{Dear Members of the Search Committee,}

I am writing to express my interest in the Neil Gehrels Prize Postdoctoral Fellowship at the Joint Space-Science Institute (JSI). My research bridges galactic astrophysics with cosmology and dark matter physics, utilizing full hydrodynamic cosmological simulations such as IllustrisTNG, EAGLE, and CAMELS to study the impact of galactic astrophysical processes on dark matter haloes. In parallel, I develop controlled numerical experiments to investigate galaxy formation and its interplay with dark matter haloes in simplified environments. These complementary approaches have equipped me with the expertise and independence to design and lead innovative research projects. \textbf{I have a clearly formulated research plan to construct a physical description of the response of dark matter haloes to galactic astrophysical processes, aiming to significantly enhance our ability to infer cosmology and dark matter physics from observations.}

Currently, I am a Senior Research Scholar at IUCAA, where I have submitted my PhD thesis under the supervision of Prof. Aseem Paranjape. My thesis focuses on the astrophysical effects of galaxy formation on dark matter haloes, emphasizing changes in radial density profiles, which are crucial to understanding observables such as rotation curves. While my primary focus has been analyzing large-scale cosmological simulations, I also develop semi-numerical experiments to study specific aspects of galaxy-halo interactions in a controlled setting. My technical expertise spans running cosmological hydrodynamic simulations using tools such as GADGET, AREPO, and MUSIC, as well as analyzing halos, galaxies, and the large-scale structure with ROCKSTAR and VELOCIraptor. Additionally, I have conducted cosmological inference from surveys like eBOSS and mock DESI data and I am also currently working in a collaboration using machine learning techniques.

Looking forward, my research will continue to probe galaxy-halo interactions across cosmic time, using a framework I developed to quantify the time-correlated effects of astrophysical feedback, particularly AGN-driven outflows, on the evolution of dark matter haloes. This research aligns well with JSI's focus on cosmology, gravitational waves, and strong gravity, particularly through its implications for understanding the relationship between baryonic feedback and large-scale structure. I am also enthusiastic about exploring potential collaborations within JSI’s interdisciplinary environment, leveraging the Institute’s partnership with NASA-Goddard Space Flight Center to connect theoretical research with cutting-edge observations.

I am deeply inspired by JSI’s dedication to fostering interdisciplinary collaboration, and I look forward to contributing to its vibrant research community. Thank you for considering my application. I have included my CV, bibliography, and research statement. I would be delighted to discuss my research and its synergy with JSI’s scientific endeavors.

\closing{Sincerely,}

\end{letter}

\end{document}
